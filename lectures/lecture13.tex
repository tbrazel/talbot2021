\documentclass{article}
\usepackage{amsmath}
\usepackage{amsthm}
\usepackage{amssymb}
\usepackage{amsfonts}
\usepackage{graphicx}
\usepackage{mathtools}
\usepackage{verbatim}
\usepackage{pgf}
\usepackage{parskip}
\usepackage{stmaryrd}
\usepackage{amscd}
\usepackage[mathscr]{euscript}
\usepackage{graphicx}
\usepackage{tikz-cd}
\usepackage{tikz}
\usetikzlibrary{decorations.pathmorphing}
\usepackage{xfrac}
\usepackage[utf8]{inputenc}
\usepackage[verbose]{wrapfig}
\usepackage{geometry} %[margin=1in]
\usepackage{fancyhdr}
\usepackage[sc,small, center]{titlesec}
\usepackage{multirow}
\usepackage{enumitem}
\usepackage{cancel} %need for strikethrough
\DeclarePairedDelimiter{\ceil}{\lceil}{\rceil}
\DeclarePairedDelimiter{\floor}{\lfloor}{\rfloor}
\usepackage{bbm}
\usepackage{hyperref}
\usepackage{todonotes}
%\usepackage{fourier} % just for the \danger [= hazard] sign

%formatting
\setitemize{noitemsep}
%\setlength{\parindent}{0pt} replaced with package 'parskip'
%\setlength{\parskip}{3mm plus 2mm minus 2mm}
%\setenumerate[0]{label=(\Alph*)}

%fixes weird spacing issue with theorem environments
\begingroup
    \makeatletter
    \@for\theoremstyle:=definition,remark,plain\do{%
        \expandafter\g@addto@macro\csname th@\theoremstyle\endcsname{%
            \addtolength\thm@preskip\parskip
            }%
        }
\endgroup

% Macros

\newcommand{\R}{{\mathbb{R}}}
\renewcommand{\C}{{\mathbb{C}}}
\newcommand{\Z}{{\mathbb{Z}}}
\newcommand{\Q}{{\mathbb{Q}}}
\newcommand{\T}{{\mathbb{T}}}
\newcommand{\nat}{{\mathbb{N}}}
\newcommand{\W}{{\mathbb{W}}}
\newcommand{\sph}{{\mathbb{S}}}
\newcommand{\E}{{\mathbb{E}}}
\newcommand{\F}{{\mathbb{F}}}
\renewcommand{\G}{{\mathbb{G}}}
\newcommand{\ep}{{\epsilon}}
\newcommand{\cat}{{\mathcal{C}}}
\newcommand{\bicat}{{\mathcal{B}}}
\newcommand{\catname}[1]{{\normalfont\textbf{#1}}}
\newcommand{\Set}{\catname{Set}}
\newcommand{\Spc}{\catname{Spc}}
\newcommand{\AbGrp}{\catname{AbGrp}}
\newcommand{\Ring}{\catname{Ring}}
\newcommand{\Aff}{\mathrm{Aff}}
\newcommand{\Grp}{\catname{Grp}}
\newcommand{\Alg}{\catname{Alg}}
\newcommand{\map}{\text{map}}
\newcommand{\id}{\text{id}}
\newcommand{\ev}{\text{ev}}
\newcommand{\op}{\text{op}}
\newcommand{\et}{\text{{\'e}t}}
\newcommand{\loops}{\Omega}
\newcommand{\Tor}{\mathrm{Tor}}

\newcommand{\del}{\partial}
\newcommand{\surj}{\twoheadrightarrow}
\renewcommand{\S}{{\mathbb{S}}}
\newcommand{\Ex}{\text{Ex}}
\newcommand{\Sp}{\text{Sp}}
\newcommand{\Ass}{\text{Ass}}
\newcommand{\CycSp}{\text{CycSp}}
\newcommand{\Fun}{\text{Fun}}
\newcommand{\Fin}{\text{Fin}}
\newcommand{\THH}{\text{THH}}
\newcommand{\inv}{\text{inv}}
\newcommand{\Mot}{\mathcal{M}}
\newcommand{\Perf}{\mathscr{P}\text{erf}}
\newcommand{\Cat}{\mathbf{Cat}}
\newcommand{\A}{\mathbb{A}}
\newcommand{\End}{\text{End}}
\newcommand{\Aut}{\text{Aut}}
\renewcommand{\Spc}{\catname{Spc}}
\newcommand{\Ch}{\catname{Ch}}
\newcommand{\Sch}{\catname{Sch}}
\newcommand{\Vect}{\catname{Vect}}
\newcommand{\Idem}{\text{Idem}}
\newcommand{\Hom}{\text{Hom}}
\DeclareMathOperator{\Ind}{Ind}
\newcommand{\Shv}{{\text{Shv}}}
\newcommand{\Pic}{\text{Pic}}
\renewcommand{\P}{\mathbb{P}}
\newcommand{\Pre}{\mathrm{Pre}}
\newcommand{\Ran}{\text{Ran}}
\newcommand{\Ob}{\mathrm{Ob}}

\newcommand{\inj}{\hookrightarrow}

\DeclareMathOperator{\Eq}{Eq}
\DeclareMathOperator*{\colim}{colim}
\DeclareMathOperator{\Spec}{Spec}
\DeclareMathOperator{\Spf}{Spf}
\DeclareMathOperator{\Ker}{Ker}
\DeclareMathOperator{\Mod}{Mod}


\newcommand{\larrow}{\longleftarrow}
\newcommand{\llarrows}{\mathrel{\substack{\textstyle\longleftarrow\\[-0.6ex]
                      \textstyle\longleftarrow}}}
\newcommand{\lllarrows}{\mathrel{\substack{\textstyle\longleftarrow\\[-0.6ex]
                      \textstyle\longleftarrow \\[-0.6ex]
                      \textstyle\longleftarrow}}}
\newcommand{\rrarrows}{\mathrel{\substack{\textstyle\longrightarrow\\[-0.6ex]
                      \textstyle\longrightarrow}}}
\newcommand{\rrrarrows}{\mathrel{\substack{\textstyle\longrightarrow\\[-0.6ex]
                      \textstyle\longrightarrow \\[-0.6ex]
                      \textstyle\longrightarrow}}}
\newcommand{\rlrarrows}{\mathrel{\substack{\textstyle\longrightarrow\\[-0.6ex]
                      \textstyle\longleftarrow \\[-0.6ex]
                      \textstyle\longrightarrow}}}
\newcommand{\rlrlrarrows}{\mathrel{\substack{\textstyle\longrightarrow\\[-0.6ex]
                      \textstyle\longleftarrow \\[-0.6ex]
                      \textstyle\longrightarrow  \\[-0.6ex]
                      \textstyle\longleftarrow  \\[-0.6ex]
                      \textstyle\longrightarrow}}}

%\renewcommand\qedsymbol{$\triangle$}

\theoremstyle{definition} \newtheorem*{defn}{Definition}
\theoremstyle{plain} \newtheorem*{prop}{Proposition}
\theoremstyle{plain} \newtheorem*{lemma}{Lemma}
\theoremstyle{plain} \newtheorem*{cor}{Corollary}
\theoremstyle{remark} \newtheorem*{ex}{Example}
\theoremstyle{remark} \newtheorem*{exs}{Examples}
\theoremstyle{remark} \newtheorem*{nonex}{Non-example}
\theoremstyle{remark} \newtheorem*{rmk}{Remark}
\theoremstyle{remark} \newtheorem*{exc}{Exercise}
\theoremstyle{remark} \newtheorem*{idea}{Idea}
\theoremstyle{remark} \newtheorem*{obs}{Observation}
\theoremstyle{plain} \newtheorem*{theorem}{Theorem}
\theoremstyle{plain} \newtheorem*{conj}{Conjecture}
\theoremstyle{remark} \newtheorem*{q}{Question}
\theoremstyle{definition} \newtheorem*{fact}{Fact}
\theoremstyle{definition} \newtheorem*{facts}{Facts}
\theoremstyle{remark} \newtheorem*{ntn}{Notation}
\theoremstyle{remark} \newtheorem*{goal}{Goal}
\theoremstyle{remark} \newtheorem*{sketch}{Sketch}
\theoremstyle{definition} \newtheorem{claim}{Claim}

\newenvironment{proofsketch}{%
  \renewcommand{\proofname}{Proof Sketch}\proof}{\endproof}

\title{E-theory of Eilenberg-MacLane spaces}
\author{Paul Van Koughnett}
\date{\today}
%\pagestyle{fancy}
%\rhead{Lucy Yang}
%\lhead{Notes}

\begin{document}
\maketitle

%\tableofcontents
\setcounter{section}{-1}

\section{Introduction}

Throughout this talk, we'll use $ \G_0 $ to denote a formal group of height $ n $ over a finite field $ k $, and $ E = E(k, \G_0) $ the associated Lubin-Tate spectrum with $ \pi_0 E = \W(k) \llbracket u_1, \ldots, u_{n} \rrbracket $. 
Let $ \G $ denote the universal deformation of $ \G_0 $ over $ \pi_0 E $. 

The goal of this talk is to prove the following
\begin{theorem}
	\begin{equation}\label{eq:pvkthm1} 
		\Spec E_0\hat B^d \Z/p^t  = Alt^{(d)}_{\G[p^t]} = \left(\Lambda^d \G[p^t] \right)^\vee 
	\end{equation}
\end{theorem}
\begin{proofsketch}[Case d = 1]
	Observe that we have the following pullback square \todo{there is a problem with using braces in tikz}
	\begin{equation}
	\begin{tikzcd}
		B\Z/p^t \ar[r] \ar[d]	& B^2 \Z \ar[d,"p^t"] \\
		* \ar[r] & B^2 \Z
	\end{tikzcd}
	\end{equation}
	which, upon taking E cohomology gives a diagram
	\begin{equation}
	\begin{tikzcd}
		\mathcal{O}_{\G p^t} = E^0 B\Z/p^t \ar[r] \ar[d]	& E^0 B^2 \Z \ar[d,"p^t"] = \mathcal{O}_G \\
		E_0 \ar[r] & E^0 B^2 \Z = \mathcal{O}_\G
	\end{tikzcd}
	\end{equation}
	where $ \mathcal{O}_\G $ is the ring of functions on the formal group $ \G $\footnote{this is true for any complex-oriented ring spectrum.} 
	Note that in fact the map on $ E $-cohomology induced by $ [p^t] $ is a flat map, so that the upper left is just an honest (i.e. underived) tensor product. 
	So what this tells us is that the upper left is the ring of functions on the $ p^t $-torsion in $ \G $. 

	So this says that $ \Spec E^0 B \Z/p^t = \G [p^t] $. 
	Since this is a finite free $ E^0 $-module, we can dualize this (this is Cartier duality) to obtain $ \Spec E_0^\vee B\Z/p^t  = \G[p^t]^\vee $. 
	This is the case $ d = 1 $, and the dual is the dual of the first exterior power. 

	Now we have to extend this to higher values of $ d $. 
	In general, there is a cup product
	\begin{equation*}
		(B \Z/p^t)^{\times d} \to B^d \Z/p^t
	\end{equation*}
	which is a skew-symmetric map. 
	Applying $ \Spec E^0 (-) $, we get a map
	\begin{equation*}
		\Spec E^0(B \Z/p^t)^{\times d} = \G[p^t]^{\times d} \to \Spec E^0 B^d \Z/p^t
	\end{equation*}
	Because the map is skew-symmetric, as long as we know that exterior powers exist, the map must factor through the exterior power. 
	Dualizing the resulting map gives the map in the theorem statement. 

	Finally, have to put together a couple of facts
	\begin{itemize}
		\item The Ravenel-Wilson calculation showed that $ K(n)_* B^d \Z/p^t $ is concentrated in even degrees and finite free of dimension $p^{t{n \choose d}}$. 

		This implies that the same is true for the $ E $-homology, i.e. that $ E^\vee_*B^d \Z/p^t $ is even and finite free of the same dimension. 

		\item We have a map between these two objects, which is an isomorphism when one tensors down to the residue field $ k $. 

		\item The idea of the proof is to (a) observe that $ E^\vee_*B^d \Z/p^t $ is finite and free over a complete local ring, (b) show that the right hand side of \ref{eq:pvkthm1eq} is also finite free over a complete local ring, (c) if they are isomorphic after tensoring down to the residue field, then they were isomorphic to begin with.\footnote{This is not \emph{quite} what's going to happen, but it's close.}
	\end{itemize}
\end{proofsketch}
A map of finite free $ E_0 $-modules that's an isomorphism over $  k $ is an isomorphism. 

Let me explain step 1. 
There's a more general statement
\begin{theorem}
	Let $ X $ be a spectrum such that $ K(n)_*(X) $ is even and finite over $ k $. 
	Then the $ E^\vee_*X $ is even and finite free over $ E_0 $. \todo{completed $ E $-homology notation?}
\end{theorem}
\begin{proof} 
	Pick some generators for the $ K(n) $ homology of $ X $: $ K(n)_*(X) = k\{x_1, \ldots, x_m \} $. 
	There's a map $ E_*(X) \to K(n)_*(X) $ which is surjective by a $ v_i $ Bockstein spectral sequence argument, so the classes $ x_i $ lift to classes $ \tilde{x}_i $in the $ E $-homology of $ X $, i.e. there is a map
	\begin{equation*}
		E_0^{\oplus m} \xrightarrow{\oplus \tilde{x}_i} E_*X \twoheadrightarrow K(n)_* X
	\end{equation*}
	The left arrow lifts to a map of spectra $ E^{\oplus m} \to E \otimes X $, which is a $ K(n) $-local isomorphism, so we are done.
\end{proof}


The hard part is defining exterior powers of these group schemes--this will occupy most of the rest of the talk. 
Before getting into it, let's make some observations. 
\begin{rmk}
	We want to think about finite flat commutative group schemes over $ R $.\footnote{I want to convince you that these are nice things to think about.} 

	You should think of this category as analogous to the category of finite abelian groups. 

	Over a field $ k $, this category is an abelian category\footnote{This is not true over a general ring, but it is still an exact category in that case.}. 
	The monomorphisms correspond to closed immersions, and the epimorphisms correspond to faithfully flat maps. 

	Every object in this category has an \emph{order} (at least when $ \Spec R $ is connected). 
	(Take any residue field of $ R $ and consider its dimension over that residue field.) 

	There's a duality\footnote{think of this as analogous to Pontrjagin duality. }

	This is equivalent to the category of finite flat, commutative, cocommutative Hopf algebras over $ R $. 
\end{rmk}

The Hopkins-Lurie paper identifies these categories using the covariant functor $ H \mapsto \Spec H^\vee $. 

There are a few simple examples worth always having in mind
\begin{itemize}
	\item constant group schemes $ \Z/p = \Spec R^{\Z/p} $
	\item multiplicative group $ \mu_p = \Spec R[x]/(x^p - 1) $
	\item additive group $ \alpha_p = \Spec R[x]/(x^p ) $
\end{itemize}
In a certain sense, these are the three most important examples. 

The only group schemes that we need to talk about for the sake of this theorem are of the form $ p^t $-torsion in some formal group. 
So we might want to restrict to `groups which look like $ p^t $-torsion.' 
We're going to be slightly more broad than that, because the exterior power operation takes you out of the world of formal groups. 

\begin{defn}
	A \emph{p-divisible group $ \G $} is a diagram\footnote{In particular, it is an Ind-object in finite group schemes} $ \G[p] \hookrightarrow \G[p^t] \hookrightarrow $ such that
	\begin{itemize}
		\item Each $ \G[p^t] $ is $ p^t $-torsion
		\item Each $ \G[p^t] $ is order $ p^{nt}$ for some fixed $ n $ which is the \emph{height} of $ \G $
		\item There is a short exact sequence
		\begin{equation*}
			0 \to \G[p] \to \G[p^t] \to \G[p^{t-1}] \to 0 
		\end{equation*}
	\end{itemize}
\end{defn}
\begin{ex}
	The constant group scheme $ \Z/p^\infty = \left\{\Z/p \to \Z/p^2 \to \cdots \right\} $. 

	$ \hat{\G}_m = \mu_{p^\infty} = \left\{\mu_p \to \mu_{p^2} \to \cdots \right\} $. 
\end{ex}
Over a field of characteristic $ p $, you can recover a formal group from its $ p $-divisible part, so passing from a formal group to its $ p $-divisible part doesn't destroy information. 

If $ \G $ is a height $ n $ formal group, the diagram of $ p $-torsion $ \G[p] \hookrightarrow \G[p^t] \hookrightarrow $ is a $ p $-divisible group. 

Note that this is a generalization of $ \hat{\G}_m $, but not $ \Z/p^\infty $ because the latter is disconnected (while a formal group must be connected). 

In general, given a formal group $ \G $ over a field $ k $, there is an exact sequence
\begin{equation*}
	0 \to \G^{form} \to \G \to \G^{\'et} \to 0 
\end{equation*}
where the étale part looks like a twist\footnote{something involving base changing to an algebraically closed field} of $ (\Z/p^\infty)^k $ for some $ k $. 

A \emph{truncated $ p $-divisible group of level $ t $} is something that can appear as the $ p^t $-torsion of some $ p $-divisible group $ \G $. 
\begin{ex}
	$ t = 1 $: $ \Z/p $ and $ \mu_p $ are examples, but $ \alpha_p $ is not. 
\end{ex}

The \emph{dimension} of a $ p $-divisible group is defined to be the dimension of its formal part. 

\begin{q}
	Why is $ \alpha_p $ not a truncated $ p $-divisible group? 
\end{q}
Answer: its Dieudonn\'e module is given by $ DM(\alpha_p) = 0 $ with $ F = V = 0 $. 
Sidebar: given a $p $-divisible group, there's a way to take a limit of the Dieudonn\'e modules of the constituent pieces to get the Dieudonn\'e module of the whole group, which would have to be a free module over $ \W(k) $. 
But there's no way to lift this to a Dieudonn\'e module which is free over $ \W(k) $. 

Sidebar: This is a pretty nonexplicit way to define a truncated $ p $-divisible group--there's a way to define it in terms of properties of $ p $-divisible groups, but I'm not going to do that. 

\section{The exterior powers}
We will define the duals first, as group schemes of alternating maps. 
Let $ G $ be a finite flat commutative group scheme over a ring $ R $ of odd order. 
\begin{defn}
	An \emph{alternating map} $ G^{\times d} \xrightarrow{f} \G_m $ is a map of schemes that is multilinear, alternating $ f(x_1, x_2, \ldots, x_d) = f(x_1, \ldots, x_i, x_{i+1}, \ldots, x_d)^{-1} $. 

	We can define the functor
	\begin{align}
		Alt_G^{(d)}: A \mapsto \{\text{alternating maps } G_A^{\times d} \to \G_{m, A} \}
	\end{align}
	This functor has a group structure inherited from the group structure on $ \G_m $. 
\end{defn}
\begin{rmk}
	This is \emph{not} the correct definition when $ p = 2 $
\end{rmk}

\begin{claim}
	$ Alt_G^{(d)} $ is representable by a group scheme. 
\end{claim}
It is easier to convince yourself of this if you drop the alternating assumption: multilinear maps are represented by $ ( G^{\otimes d})^\vee $\footnote{using the tensor product of Hopf algebras that Jan defined.} 
\begin{rmk}
	[Jacob] The tensor product is not a well-behaved object before you hom into $ \G_m $. 

	$ Alt_G^{(d)} $ is a scheme, but it's not at all obvious that it's going to be finite (indeed, it is not finite in general). 
	It's finite exactly in the situations that we're interested in. 
\end{rmk} 

\subsection{The problem when $ p = 2 $} 
An alternating map $ G \times G \xrightarrow{f} \G_m $ is one such that $ f(x, y) = f(y, x)^{-1} $. 
One way to produce such a map is to start with a two-cocycle $ c : G \times G \to \G_m $, i.e. a map satisfying $ c(x, y) c(x+y, z) = c(y, z)c(x+y, z) $. 
We can define $ f_c(x, y) = c(x,y) c(y, x)^{-1} $. 
\begin{exc}
	$ f_c $ is skew-symmetric.
\end{exc} 
So the correct definition for an alternating map is one which you can lift to a 2-cocycle:
\begin{equation*}
	Alt^{(d)}_G =\left\{\text{skew-symmetric maps which come from a 2-cocycle} \right\}
\end{equation*} 
If your group is of odd order, then very map which is alternating comes from such a $ c $. 
\begin{exc}
	If $ G $ has order $ 2m+1 $, and $ f(x, y) = f(y, x)^{-1} $, then $ f $ lifts to a cocycle defined by the formula $ c(x, y) = f(x, my) $. 
\end{exc}
However, when $ G $ is even this no longer holds. 

Once these definitions are set up, we need
 \begin{theorem}\todo{label and reorganize?}
	$ \G_0 $ a formal group over $ k $. 
	The group scheme of alternating maps $ Alg^{(d)}_{G[p^t]} = \Spec K(n)_0 B^d \Z/p^t $. 
\end{theorem} 
Recall that the way we computed this was to understand their Dieudonn\'e modules, i.e. (because we've already understood the RHS) that the LHS is a finite flat group scheme with Dieudonn\'e module $ DM(Alg^{(d)}_{G[p^t]}) = \bigwedge^d DM(\G_0[p^t])^\vee $\footnote{Question by Jan to clarify that these are exterior powers of $ \W(k) $-modules. Part of the work that goes into this is constructing a Dieudonn\'e module structure on the exterior power.}
\begin{rmk}
	This construction does \emph{not} work for $ p $-divisible groups of dimension > 1. 
\end{rmk}
Let $ G $ be a truncated $ p $-divisible group of height $ n $, dimension $ 1 $, and level  $ t $. 
Then these invariants can be seen from the Dieudonn\'e module: $ DM(G) $ is a free rank $ n $ module over $ \W(k)/p^t $, and $ DM(G)/VDM(G) \simeq k $. 
Here's why that matters: We can pick an element $ x \in DM(G) $ such that $ x, Vx, V^2x, \ldots, V^{n-1}x $ generate the $ DM(G) $. 
$$ V^n \in p \cdot DM(G) $$
Note that the Frobenius is almost determined from this information: $ Fx \in x V^{n-1} x + p \cdot DM(G) $. 
Now, $ DM(\bigwedge^d G) $ is spanned by $ a = V^{i_1} x \wedge \cdots \wedge V^{i_d}x $ for $ i_1 < i_2 < \ldots < i_d $. 
The point is that if I consider the $ \W(k) $-module spanned by these monomials, then I can define $ F $ and $ V $ such that they preserve the module\todo{submodule?}:
\begin{align*}
	V(a) &= V^{i_1 + 1}x \wedge \cdots \wedge V^{i_d + 1}x \\
	F(x \wedge V^{i_2} x \wedge \cdots \wedge V^{i_d}x ) &= F(x) \wedge V^{i_2 -1} x \wedge \cdots \wedge V^{i_d-1}x \qquad \text{ by the projection formula}
\end{align*}

\begin{theorem}
	$ Alg^{(d)}_{\G[p^t]} $ is finite flat over $ E_0 $. 
\end{theorem}

\begin{comment}
\begin{equation}
\begin{tikzcd}
	& 
\end{tikzcd}
\end{equation}
\end{comment}
\end{document}