\documentclass{article}
\usepackage{amsmath}
\usepackage{amsthm}
\usepackage{amssymb}
\usepackage{amsfonts}
\usepackage{graphicx}
\usepackage{mathtools}
\usepackage{verbatim}
\usepackage{pgf}
\usepackage{parskip}
\usepackage{stmaryrd}
\usepackage{amscd}
\usepackage[mathscr]{euscript}
\usepackage{graphicx}
\usepackage{tikz-cd}
\usepackage{tikz}
\usetikzlibrary{decorations.pathmorphing}
\usepackage{xfrac}
\usepackage[utf8]{inputenc}
\usepackage[verbose]{wrapfig}
\usepackage{geometry} %[margin=1in]
\usepackage{fancyhdr}
\usepackage[sc,small, center]{titlesec}
\usepackage{multirow}
\usepackage{enumitem}
\usepackage{cancel} %need for strikethrough
\DeclarePairedDelimiter{\ceil}{\lceil}{\rceil}
\DeclarePairedDelimiter{\floor}{\lfloor}{\rfloor}
\usepackage{bbm}
\usepackage{hyperref}
\usepackage{todonotes}
%\usepackage{fourier} % just for the \danger [= hazard] sign

%formatting
\setitemize{noitemsep}
%\setlength{\parindent}{0pt} replaced with package 'parskip'
%\setlength{\parskip}{3mm plus 2mm minus 2mm}
%\setenumerate[0]{label=(\Alph*)}

%fixes weird spacing issue with theorem environments
\begingroup
    \makeatletter
    \@for\theoremstyle:=definition,remark,plain\do{%
        \expandafter\g@addto@macro\csname th@\theoremstyle\endcsname{%
            \addtolength\thm@preskip\parskip
            }%
        }
\endgroup

% Macros

\newcommand{\R}{{\mathbb{R}}}
\renewcommand{\C}{{\mathbb{C}}}
\newcommand{\Z}{{\mathbb{Z}}}
\newcommand{\Q}{{\mathbb{Q}}}
\newcommand{\T}{{\mathbb{T}}}
\newcommand{\nat}{{\mathbb{N}}}
\newcommand{\W}{{\mathbb{W}}}
\newcommand{\sph}{{\mathbb{S}}}
\newcommand{\E}{{\mathbb{E}}}
\newcommand{\F}{{\mathbb{F}}}
\renewcommand{\G}{{\mathbb{G}}}
\newcommand{\ep}{{\epsilon}}
\newcommand{\cat}{{\mathcal{C}}}
\newcommand{\bicat}{{\mathcal{B}}}
\newcommand{\catname}[1]{{\normalfont\textbf{#1}}}
\newcommand{\Set}{\catname{Set}}
\newcommand{\Spc}{\catname{Spc}}
\newcommand{\AbGrp}{\catname{AbGrp}}
\newcommand{\Ring}{\catname{Ring}}
\newcommand{\Aff}{\mathrm{Aff}}
\newcommand{\Grp}{\catname{Grp}}
\newcommand{\Alg}{\catname{Alg}}
\newcommand{\map}{\text{map}}
\newcommand{\id}{\text{id}}
\newcommand{\ev}{\text{ev}}
\newcommand{\op}{\text{op}}
\newcommand{\et}{\text{{\'e}t}}
\newcommand{\loops}{\Omega}
\newcommand{\Tor}{\mathrm{Tor}}

\newcommand{\del}{\partial}
\newcommand{\surj}{\twoheadrightarrow}
\renewcommand{\S}{{\mathbb{S}}}
\newcommand{\Ex}{\text{Ex}}
\newcommand{\Sp}{\text{Sp}}
\newcommand{\Ass}{\text{Ass}}
\newcommand{\CycSp}{\text{CycSp}}
\newcommand{\Fun}{\text{Fun}}
\newcommand{\Fin}{\text{Fin}}
\newcommand{\THH}{\text{THH}}
\newcommand{\inv}{\text{inv}}
\newcommand{\Mot}{\mathcal{M}}
\newcommand{\Perf}{\mathscr{P}\text{erf}}
\newcommand{\Cat}{\mathbf{Cat}}
\newcommand{\A}{\mathbb{A}}
\newcommand{\End}{\text{End}}
\newcommand{\Aut}{\text{Aut}}
\renewcommand{\Spc}{\catname{Spc}}
\newcommand{\Ch}{\catname{Ch}}
\newcommand{\Sch}{\catname{Sch}}
\newcommand{\Vect}{\catname{Vect}}
\newcommand{\Idem}{\text{Idem}}
\newcommand{\Hom}{\text{Hom}}
\DeclareMathOperator{\Ind}{Ind}
\newcommand{\Shv}{{\text{Shv}}}
\newcommand{\Pic}{\text{Pic}}
\renewcommand{\P}{\mathbb{P}}
\newcommand{\Pre}{\mathrm{Pre}}
\newcommand{\Ran}{\text{Ran}}
\newcommand{\Ob}{\mathrm{Ob}}

\newcommand{\inj}{\hookrightarrow}

\DeclareMathOperator{\Eq}{Eq}
\DeclareMathOperator*{\colim}{colim}
\DeclareMathOperator{\Spec}{Spec}
\DeclareMathOperator{\Spf}{Spf}
\DeclareMathOperator{\Ker}{Ker}
\DeclareMathOperator{\Mod}{Mod}


\newcommand{\larrow}{\longleftarrow}
\newcommand{\llarrows}{\mathrel{\substack{\textstyle\longleftarrow\\[-0.6ex]
                      \textstyle\longleftarrow}}}
\newcommand{\lllarrows}{\mathrel{\substack{\textstyle\longleftarrow\\[-0.6ex]
                      \textstyle\longleftarrow \\[-0.6ex]
                      \textstyle\longleftarrow}}}
\newcommand{\rrarrows}{\mathrel{\substack{\textstyle\longrightarrow\\[-0.6ex]
                      \textstyle\longrightarrow}}}
\newcommand{\rrrarrows}{\mathrel{\substack{\textstyle\longrightarrow\\[-0.6ex]
                      \textstyle\longrightarrow \\[-0.6ex]
                      \textstyle\longrightarrow}}}
\newcommand{\rlrarrows}{\mathrel{\substack{\textstyle\longrightarrow\\[-0.6ex]
                      \textstyle\longleftarrow \\[-0.6ex]
                      \textstyle\longrightarrow}}}
\newcommand{\rlrlrarrows}{\mathrel{\substack{\textstyle\longrightarrow\\[-0.6ex]
                      \textstyle\longleftarrow \\[-0.6ex]
                      \textstyle\longrightarrow  \\[-0.6ex]
                      \textstyle\longleftarrow  \\[-0.6ex]
                      \textstyle\longrightarrow}}}

%\renewcommand\qedsymbol{$\triangle$}

\theoremstyle{definition} \newtheorem*{defn}{Definition}
\theoremstyle{plain} \newtheorem*{prop}{Proposition}
\theoremstyle{plain} \newtheorem*{lemma}{Lemma}
\theoremstyle{plain} \newtheorem*{cor}{Corollary}
\theoremstyle{remark} \newtheorem*{ex}{Example}
\theoremstyle{remark} \newtheorem*{exs}{Examples}
\theoremstyle{remark} \newtheorem*{nonex}{Non-example}
\theoremstyle{remark} \newtheorem*{rmk}{Remark}
\theoremstyle{remark} \newtheorem*{exc}{Exercise}
\theoremstyle{remark} \newtheorem*{idea}{Idea}
\theoremstyle{remark} \newtheorem*{obs}{Observation}
\theoremstyle{plain} \newtheorem*{theorem}{Theorem}
\theoremstyle{plain} \newtheorem*{conj}{Conjecture}
\theoremstyle{remark} \newtheorem*{q}{Question}
\theoremstyle{definition} \newtheorem*{fact}{Fact}
\theoremstyle{definition} \newtheorem*{facts}{Facts}
\theoremstyle{remark} \newtheorem*{ntn}{Notation}
\theoremstyle{remark} \newtheorem*{goal}{Goal}
\theoremstyle{remark} \newtheorem*{sketch}{Sketch}
\theoremstyle{definition} \newtheorem{claim}{Claim}

\newenvironment{proofsketch}{%
  \renewcommand{\proofname}{Proof Sketch}\proof}{\endproof}

\title{The Global section functor}
\author{Adela Zhang}
\date{\today}
%\pagestyle{fancy}
%\rhead{Lucy Yang}
%\lhead{Notes}

\begin{document}
\maketitle

\section{Introduction}

Today I'll try to bring us back to one of the motivations for this topic, which is is representation theory. 

We know that for any space $ X $, there is an equivalence
\begin{equation*}
	\{\text{Local systems on }X\} \simeq \{\text{Representations of } \loops X \}
\end{equation*}

Given a map $ f: X \to * $, we get a pair of adjoints $ f_*: \Sp_{K(n)}^X \to \Sp_{K(n)} f^*$. 

Denote the $ K (n)$-local sphere by $ \S = \S_{K(n)} $. 
Then $ \Sp_{K(n)}^X \simeq LMod_{\underline{\S}_X}(\Sp_{K(n)}^X) \to LMod_{f_*\underline{\S}_X}(\Sp_{K(n)}) $ where $ \S_X $ is the constant sheaf. 
\begin{equation}
\begin{tikzcd}
	\Sp_{K(n)}^X \ar[r, bend right = 20, "f_*"] & \ar[l, bend right=20, "f^*"] \Sp_{K(n)} \\
	 LMod_{\underline{\S}_X}(\Sp_{K(n)}^X) \ar[u, equals] \ar[r, bend right = 20, "F"] & LMod_{f_*\underline{\S}_X}(\Sp_{K(n)}) \ar[u, "\text{free}"] \ar[l, bend right = 20, "G"]
\end{tikzcd}
\end{equation}
Note that the pushforward of the constant sheaf is equivalent to the function spectrum $ f_*\underline{\S}_X \simeq C(X; \S) $. \todo{fix}

Now the above picture holds for any space $ X $. 
Now we can put some conditions on $ X $ and use our knowledge of $ K(n) $-local spectra:

If $ X $ is $ \pi $-finite, then $ G $ is fully faithful. 

Furthermore, if $ X $ is $ n $-truncated and $ p $-finite\todo{can't tell from the video where exactly this should belong?}, then $ G $ is an equivalence. 

Note that $ \Sp_{K(n)}^X \simeq LMod_{\loops X}(\Sp_{K(n)}) $, so what we're saying is that when $ X $ satisfies the assumptions above, all representations are induced from the trivial representation.\footnote{I think Tomer is saying that this all holds with $ T(n) $-local instead of $ K(n) $-local.}

Tomer pointed out yesterday that if $ X $ is $ \pi $-finite and $ (n+2) $-connective, then the pullback functor $ f^* $ is an equivalence--this says that $ K(n) $-local Morava $ E $-theory cannot distinguish between highly-connected spaces and a point. 

The goal of my talk is to prove the various equivalences stated above. 

\section{}
We'll start by defining $ F, G $ from above in a more general situation. 

Fix a map of spaces $ f:X \to Y $. 
Recall that $ \Sp_{K(n)} $ is a symmetric monoidal category. 
Therefore, local systems $ \Sp_{K(n)}^X $ inherit a symmetric monoidal structure given by the pointwise tensor product. 
Now let's observe that the pullback functor $ f^* $ is a symmetric monoidal functor--in particular, it is oplax symmetric monoidal. 
From this it follows that the pushforward $ f_* $ is lax symmetric monoidal, and it makes sense to pushforward algebra objects and modules over such. \todo{reword}

For $ A \in \Alg\left(\Sp_{K(n)}^X\right) $, the pushforward $ f_*A \in \Alg\left(\Sp_{K(n)}^Y\right) $, and if $ M \in \LMod_A $, then $ f_*M \in \LMod_{f_*A} $. 

Thus we get a pushforward functor 
\begin{equation*}
	F: LMod_A(\Sp^X_{K(n)}) \to LMod_{f_*A}(\Sp^Y_{K(n)})
\end{equation*}

The first result I will attempt to sketch is
\begin{theorem}\label{thm:adelathm1}
	\begin{enumerate}
		\item If the homotopy fibers of $ f $ are $ \pi $-finite, then $ F $ admits a left adjoint $ G $ and $ G $ is fully faithful. 

		\item If the homotopy fibers of $ f $ are $ p $-finite and $ n $-truncated, then $ G $ is an equivalence. 
	\end{enumerate}
\end{theorem}






\begin{comment}
\begin{equation}
\begin{tikzcd}
	& 
\end{tikzcd}
\end{equation}
\end{comment}
\end{document}
