\renewcommand{\thespeaker}{Peter Haine}
\renewcommand{\thetitle}{Character theory via tempered cohomology}
\section{\thetitle~(\thespeaker)}
\providecommand{\Supp}{\text{Supp}}

\textbf{Plan}:
\begin{enumerate}
    \item Motivation from character theory
    \item How the statements in (1) relate to tempered cohomology
    \item Orientations
\end{enumerate}

\textbf{Key examples}:
\begin{enumerate}
    \item $\KU_{\mu_\mathcal{P}^\infty}$
    \item $A=E_n$ and $G = \mathbb{G}^\text{Quillen}$
\end{enumerate}

\subsection{Character theory of finite groups}

Given a finite group $G$, there is an isomorphism
\begin{align*}
    \Rep(G) &\xto{\chi} \left\{ \text{class function} G \to \mathbb{C} \right\},
\end{align*}
which is an isomorphism after applying $- \otimes_\mathbb{Z} \mathbb{C}$. We know that $\Rep(G) = \KU_G^0(\ast)$. Since the class functions are conjugation-invariant, they are functions on an adjoint quotient of $G$ by itself to the complex numbers. We think about this as $H^0(-, \mathbb{C})$ of a quotient of $G$ by its conjugation. That is, we think about the set of class functions as
\begin{align*}
    H^0 \left( G_{hG}, \mathbb{C} \right).
\end{align*}

\begin{theorem} If $X$ is a finite $G$-space, write $\amalg_{g\in G} X^g$ for the space with $G$-action given by conjugation on the indexing set $G$ and residual actions on the fixed point spaces $X^g := X^{\left\langle g \right\rangle}$ for each $g$. Then the \textit{equivariant Chern character} defines an isomorphism
\begin{align*}
    \mathbb{C} \otimes_\mathbb{Z} \KU_G^0(X) \xto{\sim} H^{\text{even}} \left( \left( \amalg_g X^g \right)_{hG}, \mathbb{C} \right).
\end{align*}
\end{theorem}

\textbf{Question}: What about the $p$-complete $K$-theory analogue of this statement?

We have that $G \cong \Hom(\mathbb{Z},G)$, so we are going to replace this by $\Hom \left( \mathbb{Z}_p, G \right)$. This latter group is finite, so they are factoring through some subgroup of finite order. This is
\begin{align*}
    \Hom \left( \Z_p, G \right) &= \left\{ g\in G \colon g^{p^k} = 1 \text{ for } k \gg 0 \right\}.
\end{align*}
These elements are called $p$\textit{-singular elements}.

\begin{theorem} Fix a prime $p$ and an embedding $\mathbb{Z}_p \hookto \mathbb{C}$. Then there is a canonical isomorphism
\begin{align*}
    \mathbb{C} \otimes_{\mathbb{Z}_p} \KU_p^{\wedge,0} \left( X_{hG} \right) \xto{\sim} H^\text{even} \left( \left( \amalg_{\alpha:\mathbb{Z}_p \to G} X^{\im(\alpha)} \right)_{hG}, \mathbb{C} \right).
\end{align*}
On the left we have height 1, while on the right we have height zero.
\end{theorem}

\textbf{Question}: What about relating height $n$ and height $0$, or more generally height $m$ for $m\le n$. Relating height $n$ with height zero is Hopkins--Ravenel character theory. Height $n$ and $m$ is due to Stapleton.

\textbf{Setup}:
\begin{itemize}
    \item Let $k$ be a perfect field of characteristic $p$, let $\hat{\mathbb{G}}_0$ be a height $n$ formal group, with $\hat{\mathbb{G}}$, the identity component of a $p$-divisible group, its universal deformation
    \item $R$ a Lubin--Tate ring
    \item $C_0= R$-algebra classifying isos $\mathbb{G} \cong \left( \mathbb{Q}_p/\mathbb{Z}_p \right)^\wedge$
    \item $E$ the Lubin--Tate ring of $\hat{\mathbb{G}}_0$
\end{itemize}

\begin{customenvironment}{HKR theorem} Let $G$ be a finite group and $X$ a finite $G$-space:
\begin{align*}
     C_0 \otimes_R E^\ast \left( X_{hG} \right) \xto{\sim}  C_0 \otimes_{R_\mathbb{Q}} E_\mathbb{Q}^\ast \left( \left( \amalg_{\alpha: \mathbb{Z}_p \to \mathbb{C}} X^{\im(\alpha)} \right)_{hG} \right) 
\end{align*}

\end{customenvironment}

\textbf{Key steps}:
\begin{enumerate}
    \item Understand what happens to tempered cohomology if we use something like $A_{\mathbb{G}_0 \oplus \left(\mathbb{Q}_p/\mathbb{Z}_p\right)^n}$
    \item $B \otimes_A A_G^{\mathbb{X}}$ versus $B_{\mathbb{G}}^{\mathbb{X}}$
\end{enumerate}

\subsection{Formal loop spaces}

Recall $\mathcal{T} \subseteq \Spc$ denotes the full subcategory on objects $BG$ with $G\in \Ab^\text{fin}$, $\O \Spc = \PSh(\mathcal{T})$

\begin{definition} For $\mathbb{X} \in \Ospc$, and $\Lambda$ a torsion abelian subgroup, then the formal loop space $\mathcal{L}^\wedge(\mathbb{X})$ is defined as
\begin{align*}
    \mathcal{T}^\op &\to \Spc \\
    T &\mapsto \colim_{\substack{\Lambda_0 \subseteq \Lambda \\ \text{finite}}} \left(\mathbb{X}^{T \times B\Lambda_0^\vee}\togjt)
\end{align*}
\end{definition}

\begin{customenvironment}{Computation} If $X$ is a $G$-space, we can take the formal loop space on its orbispace quotient
\begin{align*}
    \mathcal{L}^\wedge \left( X // G \right) \simeq \left(\coprod_{\alpha: \Lambda^\vee \to G} X^{\im(\alpha)}\right) // G.
\end{align*}
\end{customenvironment}

If $X$ is $\pi$-finite, then $\mathcal{L}^{\mathbb{Q}/\mathbb{Z}}(X) \simeq \mathcal{L}(X)$. Similarly if $X$ is $p$-finite, then $\mathcal{L}^{\mathbb{Q}_p / \mathbb{Z}_p}(X) \simeq \mathcal{L}(X)$.

\textbf{Recall} That $\left( \mathbb{Q}_p/ \mathbb{Z}_p \right)^\vee \cong \mathbb{Z}_p$. Similarly $\left( \mathbb{Q}/\mathbb{Z} \right)^\vee \cong \hat{\mathbb{Z}}$.

\begin{note} If $X\in \Spc$, we have $X^{(-)} \in \OSpc$, There's a map
\begin{align*}
    \left| \mathcal{L}^\wedge \left( X^{(-)} \right) \right| \simeq \colim_{\Lambda_0 \subseteq \Lambda} X^{B\Lambda_0^\vee} \to X^{B\Lambda^\vee},
\end{align*}
where in this last step we are ignoring the profinite topology. That gives you a natural map
\begin{align*}
    \mathcal{L}^\wedge X^{-)} \to \left( X^{B\Lambda ^\vee }\right)
\end{align*}
\end{note}

\begin{definition} We say an abelian group $\Lambda$ is a \textit{collatice} if
\begin{enumerate}
    \item $\Lambda$ is torsion
    \item For all $n$, the multiplication by $n$ map $\Lambda \xto{n\cdot-} \Lambda$ is a surjection with finite kernel.
\end{enumerate}
\end{definition}

Equivalent, if for all $p$, we have that $\Lambda_{(p)} \cong \left( \mathbb{Q}_p / \mathbb{Z}_p \right)^{h_p}$ for various heights.

\begin{proposition} Let $X$ be a $\pi$-finite space, and $\Lambda$ is a colattice, then
\begin{align*}
   \mathcal{L}^\wedge \left( X^{(-)} \right) \xto{\sim} \left( X^{B\Lambda^\vee} \right)^{(-)}
\end{align*}
\end{proposition}

\begin{customenvironment}{Character isomorphism} Let $\mathbb{G}_0$ be a preoriented $\mathcal{P}$-divisible group, $\Lambda$ a colattice, and $\mathbb{G} := \mathbb{G}_0 \oplus \underline{\Lambda}$. Then for any $\mathbb{X} \in \OSpc$, we have an isomorphism
\begin{align*}
    \chi : A^\ast_{\mathbb{G}} \simeq A_{G_0}^{\mathcal{L}^\e}.
\end{align*}
\end{customenvironment}

\begin{corollary} In the same setup, for $G$ a finite group and $X$ a $G$-spce, we have
\begin{align*}
    \chi : A_G^\ast(X // G) \simeq \mathbb{A}_{\mathbb{G}_0}^\ast \left( \left( \amalg_{\alpha : \Lambda^\vee \to \mathbb{C}} X^{\im(\alpha)} \right)  // G\right).
\end{align*}
\end{corollary}

\subsection{Chern character via tempered cohomology}

\textbf{Setup}: Write $\KU_\mathbb{C} := \mathbb{C} \otimes \KU$. Over $\mathbb{C}$ there is an exponential of $p$-divisible groups
\begin{align*}
    \exp : \underline{\mathbb{Q}/\mathbb{Z}} & \xto{\sim} \mu_{\mathcal{P}^\infty} \\
    \lambda &\mapsto \exp \left( 2\pi i \lambda \right).
\end{align*}
We have that
\begin{align*}
    \left( \KU_\mathbb{C} \right)_{\mu_{\mathcal{P}^\infty}}^{\mathbb{X}} \simeq \left( \KU_\mathbb{C} \right)_{\underline{\mathbb{Q}/\mathbb{Z}}}^{\mathbb{X}} \simeq \KU_\mathbb{C}^{\left| \mathcal{L}^{\mathbb{Q}/\mathbb{Z}(\mathbb{X})} \right|}
\end{align*}
Precomposing with a map from ordinary $\KU$ we get a composite we call the Chern character
\begin{align*}
    \Ch: \KU_{\mu_{\mathcal{P}^\infty}}^\ast \left( \mathbb{X} \right) \to H^\ast \left( \left| \mathcal{L}^{\mathbb{Q}/\mathbb{Z}}(\mathbb{X}) \right|, \mathbb{C} \right)[\beta^{-1}].
\end{align*}

We haven't yet used Atiyah--Segal completion in tempered cohomology. We will introduce this, and this is when the base change theorem will hold.

\subsection{Orientations}

\textbf{Goal}: Construct a map
\begin{align*}
    \zeta : A_\mathbb{G}^{\mathbb{X}} \to A^{\left| \mathbb{X} \right|}.
\end{align*}

\begin{fact} $A_\mathbb{G}^{\underline{X}} \simeq A^X$.
\end{fact}

So it suffices, for constructing the map above, to construct a map $\underline{\left| \mathbb{X} \right|} \to \mathbb{X}$. But we already have one, this is just the counit. So the Atiyah--Segal completion map is just $\zeta$ above using the counit.

\begin{note} The point $\ast \to BG$ determines a surjection $A_{\mathbb{G}}^0 (BG) \to A_\mathbb{G}^0(\ast) = \pi_0 A$. We will write $I_G$ for the kernel of this map.
\end{note}

\begin{definition} Let $A$ be an $E_\infty$-ring, $\mathbb{G}$ a preoriented $\mathcal{P}$-divisible group over $A$. Then we say that $\mathbb{G}$ is \textit{oriented} if:
\begin{enumerate}
    \item For all $p$, the Atiyah--Segal comparison map
    \begin{align*}
        \zeta: A_\mathbb{G}^{BC_p} \to A^{BC_p}
    \end{align*}
    exhibits $A^{BC_p}$ as the $I_{C_p}$-completion of the source.

    \item The Tate construction $A^{tC_p}$ is $I_{C_p}$-local as an $A_\mathbb{G}^{BC_p}$-module.
\end{enumerate}
\end{definition}

\begin{theorem} (Base change) If $\mathbb{G}$ is an oriented $\mathcal{P}$-divisible group over $A$, then for every $\pi$-finite space $X$, given a map $f: A \to B$, we have that
\begin{align*}
    B \otimes_A A_\mathbb{G}^X \xto{\sim} B_\mathbb{G}^X
\end{align*}
is an equivalence.
\end{theorem}

